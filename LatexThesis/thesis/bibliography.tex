\begin{thebibliography}{100}
    \bibitem{FARE}Food Allergy Research and Education, “Preventing Food Allergies: Early Interventions | FARE,” \textit{www.foodallergy.org.} https://www.foodallergy.org/research-innovation/accelerating-innovation/early-introduction-and-food-allergy-prevention
    
    \bibitem{Brown} D. Brown et al., “Food allergy-related bullying and associated peer dynamics among black and white children in the forward study,” \textit{Annals of Allergy, Asthma and Immunology, vol. 126, no. 3}, 2021. doi:10.1016/j.anai.2020.10.013
    
    \bibitem{AAAAI} American Academy of Allergy Asthma and Immunology, “The Current State of Oral Immunotherapy (OIT) for the Treatment of Food Allergy,” \textit{Aaaai.org}, 2020. https://www.aaaai.org/Tools-for-the-Public/Conditions-Library/Allergies/The-Current-State-of-Oral-Immunotherapy 

    \bibitem{ACAAI} The American College of Allergy, Asthma and Immunology, “Food Allergy Avoidance,” ACAAI Public Website. https://acaai.org/allergies/management-treatment/living-with-allergies/food-allergy-avoidance/

    \bibitem{Bilucaglia} M. Bilucaglia et al., "Looking through blue glasses: bioelectrical measures to assess the awakening after a calm situation," \textit{2019 41st Annual International Conference of the IEEE Engineering in Medicine and Biology Society (EMBC)}, Berlin, Germany, 2019, pp. 526-529, doi: 10.1109/EMBC.2019.8856486.

    \bibitem{Blumchen} K. Blumchen et al., “Post hoc analysis examining symptom severity reduction and symptom absence during food challenges in individuals who underwent oral immunotherapy for peanut allergy: results from three trials,” Allergy, Asthma, and Clinical Immunology: Official Journal of the Canadian Society of Allergy and Clinical Immunology, vol. 19, no. 1, p. 21, Mar. 2023, doi: https://doi.org/10.1186/s13223-023-00757-8.

    \bibitem{Wanniang} N. Wanniang et al., “Immune signatures predicting the clinical outcome of peanut oral immunotherapy: where we stand,” Frontiers in Allergy, vol. 4, Oct. 2023, doi: https://doi.org/10.3389/falgy.2023.1270344.

    \bibitem{Nairn} S. A. Nairn, “Creating an (ethical) epistemic space for the normalization of clinical and ‘real food’ oral immunotherapy for food allergy,” Health: An Interdisciplinary Journal for the Social Study of Health, Illness and Medicine, p. 136345932211096, Jul. 2022, doi: https://doi.org/10.1177/13634593221109679.

    \bibitem{Dominguez} T. Dominguez, “Food Allergy, Oral Food Challenges, and Oral Immunotherapy,” Physician Assistant Clinics, vol. 8, no. 4, pp. 675–684, Oct. 2023, doi: https://doi.org/10.1016/j.cpha.2023.05.003.

    \bibitem{Gilbert} M. Gilbert T. Chua et al., “Real-world safety and effectiveness analysis of low-dose preschool sesame oral immunotherapy,” Journal of Allergy and Clinical Immunology: Global, vol. 3, no. 1, p. 100171, Feb. 2024, Accessed: Nov. 29, 2023. [Online]. Available: https://doaj.org/article/419f5832508a49019ca3476a94bd4b3d

    \bibitem{Bird} M. D. J. Andrew Bird et al., “Long-term safety and immunologic outcomes of daily oral immunotherapy for peanut allergy,” Journal of Allergy and Clinical Immunology: Global, vol. 2, no. 3, p. 100120, Aug. 2023, Accessed: Nov. 29, 2023. [Online]. Available: https://doaj.org/article/880df47945f042aeb9a2cf79f96c1241

    \bibitem{UML} “About the Unified Modeling Language Specification Version 2.5.1,” www.omg.org. https://www.omg.org/spec/UML/

    \bibitem{IEEEStd} “IEEE Recommended Practice for Software Requirements Specifications,” IEEE Std 830-1998, pp. 1–40, Oct. 1998, doi: https://doi.org/10.1109/IEEESTD.1998.88286.

    \bibitem{ISO} ISO, “ISO/IEC 25010:2011,” ISO, 2011. https://www.iso.org/standard/35733.html 

    \bibitem{HIG} Apple, “Human Interface Guidelines - Design - Apple Developer,” Apple.com, 2019. https://developer.apple.com/design/human-interface-guidelines/

    \bibitem{AppStore} Apple, “App Store Review Guidelines - Apple Developer,” Apple.com, 2019. https://developer.apple.com/app-store/review/guidelines/

    \bibitem{AppleAccessibility} Apple, “Accessibility - Apple Developer,” Apple.com, 2019. https://developer.apple.com/accessibility/

    \bibitem{HIPPA} H. Health, “U.S. Department of Health and Human Services,” HHS.gov, 2019. https://www.hhs.gov

    \bibitem{HealthKit} Apple, “HealthKit | Apple Developer Documentation,” developer.apple.com. https://developer.apple.com/documentation/healthkit

    \bibitem{CareKit} Apple, “CareKit - Apple Developer,” developer.apple.com. https://developer.apple.com/carekit/

    \bibitem{SwiftCharts} [17]Apple, “Apple Developer Documentation,” developer.apple.com. https://developer.apple.com/documentation/charts

    \bibitem{SwiftData} Apple, “SwiftData,” Apple Developer Documentation. https://developer.apple.com/documentation/swiftdata

    \bibitem{Kant} Kant, Immanuel. "Groundwork of the Metaphysics of Morals." Cambridge University Press, 1998.

    \bibitem{Mill} Mill, John Stuart. "Utilitarianism." Edited by George Sher. Hackett Publishing Company, 2001.

    \bibitem{Nussbaum} Nussbaum, Martha C. "Creating Capabilities: The Human Development Approach." Harvard University Press, 2011.

    \bibitem{ACM} ACM Code 2018 Task Force. “ACM Code of Ethics and Professional Conduct.” ACM: Association for Computing Machinery, 2018, https://www.acm.org/code-of-ethics. 

\end{thebibliography}