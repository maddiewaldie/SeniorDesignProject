\chapter{Societal Issues}

When developing a project like this, it is crucial to consider a broad spectrum of societal issues to ensure responsible and ethical innovation. This chapter explores various dimensions, including ethical, social, political, economic, health and safety, manufacturability, sustainability, environmental impact, usability, lifelong learning, and compassion.

\section{Ethical}

Ethical considerations are fundamental to the creation and utilization of any medical application. The oral immunotherapy application adheres to strict ethical standards in terms of user privacy, data security, and informed consent. The collection and handling of user data prioritize confidentiality and comply with relevant data protection regulations. Additionally, the application does not promote any form of discrimination or bias, ensuring fair and equitable access to its benefits. A full, more detailed ethical analysis of the application can be found in Chapter \ref{section:ethics}.

\section{Social}

The social dimension of the oral immunotherapy application focuses on its broader impact on individuals and the community. While the app does not include forums or community features, it still contributes significantly to the social aspect of oral immunotherapy management. By providing a tool for individuals to effectively manage their treatment, the app promotes a sense of autonomy and self-efficacy. This empowerment can lead to improved social interactions and a more active engagement with one's personal health journey. Moreover, the app facilitates better communication between patients and healthcare providers, fostering a collaborative and supportive relationship within the broader healthcare system. The social impact of the application extends beyond user interactions, playing a vital role in enhancing the overall quality of life for individuals undergoing oral immunotherapy.

\section{Political}

Political considerations encompass the alignment of the oral immunotherapy app with healthcare policies and regulations. The application complies with relevant medical standards and guidelines, contributing to the political goal of enhancing patient care and treatment adherence.

\section{Economic}

Economic factors focus on the financial implications of developing and adopting the oral immunotherapy app. The application is designed to be freely accessible to a wide range of users. Additionally, by promoting self-management and reducing healthcare visits, the app has the potential to alleviate economic burdens associated with oral immunotherapy treatments.

\section{Health and Safety}

Health and safety considerations are paramount in the development of a medical application. The app prioritizes user safety by providing accurate information, timely reminders, and emergency protocols. The app encourages responsible self-management while emphasizing the importance of consulting healthcare professionals for critical decisions. Finally, the application complies with Apple Health's constraints and standards, ensuring the data is stored safely and accurately.

\section{Manufacturability}

Manufacturability considerations revolve around the scalability and efficiency of the app's production and distribution. The oral immunotherapy app is designed to be easily scalable, with updates and improvements seamlessly integrated into the user experience. The manufacturing process, in this context, involves software development practices that prioritize reliability, security, and ease of deployment.

\section{Sustainability}

Sustainability considerations encompass the long-term viability and relevance of the oral immunotherapy app. The application is built with a modular and adaptable architecture, allowing for continuous updates and enhancements to meet evolving healthcare needs. Sustainable development practices ensure the app remains effective and relevant in the ever-changing landscape of oral immunotherapy.

\section{Environmental Impact}

The environmental impact of the oral immunotherapy app is predominantly associated with its digital nature, minimizing physical waste and resource consumption. By promoting digital interactions and reducing the need for physical documentation, the app aligns with environmentally conscious practices.

\section{Usability}

Usability is a crucial societal factor that influences the app's acceptance and effectiveness. The oral immunotherapy app prioritizes a user-friendly interface, intuitive navigation, and accessibility features to cater to a diverse user base. Enhancing usability contributes to the app's overall positive societal impact by ensuring broad accessibility and inclusivity.

\section{Lifelong learning}

Lifelong learning considerations emphasize the app's role in providing continuous education and support to users throughout their oral immunotherapy journey. The application incorporates educational resources, updates, and relevant information to foster ongoing learning and empowerment for users, promoting informed decision-making and self-management.

\section{Compassion}

Compassion is woven into the fabric of the oral immunotherapy app, recognizing the challenges individuals face in managing their oral immunotherapy treatments. The app is designed not only to provide practical support but also to empathize with users, acknowledging the emotional and physical aspects of their journey. Through features like personalized reminders and motivational messages, the app aims to instill a sense of compassion and understanding in its interaction with users.