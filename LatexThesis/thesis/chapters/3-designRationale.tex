\chapter{Research}

While I know a lot about OIT from my own experiences, it was important that I thoroughly researched the topic before creating my initial design – especially because OIT is a new, emerging treatment and isn't standardized, so it's different from clinic to clinic. This chapter outlines my research on OIT, shedding light on its definition and mechanisms, as well as the initial advice I received from allergists and immunologists in the field.

\section{Research}

\subsection{Background}
Oral Immunotherapy has exhibited remarkable efficacy in addressing a spectrum of food allergies, including peanuts, tree nuts, milk, and eggs. The impact on patients' quality of life is profound, offering hope and tangible relief. As revealed by recent studies, the success of OIT is particularly notable in young patients, especially those under three years old. For example, a study found that young patients demonstrate the highest success rates in OIT, showcasing the treatment's potential for transformative outcomes in this age group \cite{Blumchen}.

The mechanistic underpinnings of OIT unfold within the intricate interactions of the immune system. Research indicates that carefully administered doses of allergens through OIT orchestrate a controlled immune response \cite{Nairn}. This calibrated exposure triggers the production of regulatory T cells and other immune modulators, facilitating a shift from hypersensitivity to a more tolerant state. Such immune modulation is crucial for the success of OIT, and understanding these processes informs the app design to ensure patients comprehend the science behind their treatment. 

\subsection{The Patient's Journey}

The OIT patient journey is a nuanced process, beginning with a meticulous selection phase that takes into account factors such as age, allergy severity, and overall health. Recent research has highlighted the critical role of comprehensive assessments and diagnosis in guiding a tailored treatment plan. For instance, a study found that the success of OIT is intricately linked to understanding the specific allergens triggering reactions in individual patients \cite{Dominguez}. This research underscores the importance of incorporating detailed allergy testing and evaluations into the app to guide users through a personalized OIT journey.

The carefully designed OIT protocol involves a gradual introduction of allergens in controlled doses, with customization based on individual needs. Data from ongoing studies emphasizes that adapting the frequency and intensity of allergen doses to individual responses is essential for the efficacy of OIT. Incorporating this insight into the app ensures that users have a clear understanding of the gradual exposure process, fostering adherence and positive engagement.

\subsection{Predosing Medications and Precautions}

Predosing medications play a pivotal role in preparing patients for controlled allergen exposure during OIT. Recent findings, as seen in various clinical studies, underscore the importance of this precautionary step. For example, research suggests that administering medications, such as antihistamines, mitigates the risk of immediate allergic reactions, enhancing the overall safety profile of OIT \cite{Gilbert}. This crucial insight informs the app's design to include features that remind users about predosing medications and educate them on their significance in ensuring a smooth OIT progression.

The close monitoring of patient responses to predosing medications, as supported by research, further emphasizes the adaptability of treatment plans to individual sensitivities. Integrating tools in the app that allow users to track and report their responses ensures a more personalized OIT experience, aligning with the variability observed in patient reactions.

\subsection{Adherence to Strict Rules and Lifestyle Adjustments}

The path to successful Oral Immunotherapy involves a commitment to adherence to strict rules and lifestyle adjustments. Patients partaking in OIT are often advised to refrain from vigorous exercise after each session and abstain from alcohol during the treatment period. Research has demonstrated that these seemingly restrictive measures are integral to the safety and efficacy of the treatment \cite{Dominguez}.

For instance, ongoing studies have shown that adherence to guidelines significantly influences the success of OIT, reducing the risk of adverse reactions. This underscores the personalized nature of OIT, emphasizing the need for patients to actively engage in their own care. The app design takes inspiration from this research, incorporating features that not only educate users on the importance of adherence but also provide tools for tracking and maintaining lifestyle adjustments.

\subsection{Highly Personalized Treatment Plans}

A key takeaway from the research is the highly personalized nature of OIT treatment plans. Recent studies have emphasized that the success of OIT lies in recognizing the uniqueness of each patient's journey. Treatment plans are meticulously crafted based on factors such as age, medical history, specific allergens, and overall health. This tailored approach allows for adjustments in allergen dose frequency and intensity, ensuring effective and well-tolerated progress.

For example, ongoing research has shown that individualized treatment plans significantly contribute to positive outcomes in OIT. Patients progress through therapy at a pace that is both effective and well-tolerated, reducing the risk of adverse reactions. The individualized nature of OIT reflects its commitment to providing bespoke solutions, acknowledging the diversity of allergic conditions and accommodating variations in the treatment process. The app, inspired by this research, prioritizes customization, offering users a personalized roadmap for their OIT journey.

\section{Consultation with Immunologists}

After thoroughly researching OIT, the natural next step in my research and design process was to bring my app idea to doctors, and garner their advice. I collaborated with top immunologists from The Children's Hospital of Philadelphia, Mt. Sinai Hospital, and Stanford Medicine to ensure that my application aligns with expert advice and industry best practices. Here are some key pieces of advice I received:

\begin{itemize}
    \item \textbf{Pop Up to Seek Medical Care: }Immunologists recommended displaying a pop-up notification encouraging users to ``Call your doctor" or ``Call 911" if any aspect of their symptoms raises concerns.
    \item \textbf{Engagement through Notifications: }Immunologists emphasized the importance of subtle reminders, such as ``Did you take your dose today?" These notifications not only promote adherence to treatment plans but also serve as a gentle nudge towards consistent health practices.
    \item \textbf{Emergency Preparedness: }Immunologists advised incorporating an innovative feature allowing users to upload images of their emergency action plans to the 'Resources' section. This practical addition ensures that critical information is readily accessible, fostering a sense of empowerment in navigating unforeseen health scenarios.
    \item \textbf{Educational Prompts:} In the realm of education, immunologists recommended maintaining a delicate balance to avoid offering explicit medical advice. Instead, the app should gently prompt users with thought-provoking questions like, ``Did you ask your doctor about this? Or that?" This approach encourages informed discussions with healthcare providers, fostering a proactive and engaged patient community.
    \item \textbf{Autoinjector Management: }Recognizing the prevalence of autoinjectors in allergy management, immunologists suggested a dedicated section providing functionality to check the expiration status of these crucial devices. This integration aligns with the app's commitment to supporting users in maintaining the efficacy of their medical tools.
    \item \textbf{Individualization: }Immunologists highlighted the importance of individualization, guiding the development process with an unwavering commitment to recognizing the distinct needs of every user. This approach acknowledges the nuanced nature of healthcare management.
    \item \textbf{Gamification Elements: }To foster user motivation, immunologists recommended embracing an innovative approach by incorporating gamification elements. This is something I hadn't thought of adding, but if I had time, would love to incorporate it into my application.
\end{itemize}

In essence, the collaborative efforts with renowned immunologists have shaped an app that transcends conventional health management tools. By embracing flexibility, individualization, and innovative features, this application aspires to be not just a companion in healthcare but a personalized guide, empowering users to navigate their health journeys with confidence and informed decision-making.