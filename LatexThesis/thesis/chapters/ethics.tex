\chapter{An Ethical Analysis}
\label{section:ethics}

Ethical considerations play a crucial role in engineering projects, particularly in the realm of healthcare and medical technology. Engineering endeavors, especially those aimed at improving healthcare outcomes, inherently involve decisions that impact individuals' well-being, safety, and autonomy. Therefore, in this paper, I seek to explore these ethical dimensions within the context of OIT application development, explaining how I designed a solution that not only enhances patient experience and treatment outcomes but also aligns with ethical principles governing healthcare innovation.

The following sections will outline the ethical justification for my project, identify significant ethical issues, provide reasons for decisions and actions, draw on ethical resources, and conclude with reflections on the ethical implications of the project for the broader healthcare landscape. Throughout this paper, I will draw upon established ethical frameworks, engineering codes of ethics, and additional ethical concepts to ensure a comprehensive analysis of the ethical considerations surrounding the development and implementation of the OIT application.

\section{Ethical Justification for the Project}

The very foundation of my project is built upon ethics – particularly the ethical principle of beneficence, which underscores the moral obligation to act for the benefit of others, promoting their well-being and preventing harm. 

As philosopher Immanuel Kant eloquently stated, ``Act in such a way that you treat humanity, whether in your own person or in the person of any other, never merely as a means to an end, but always at the same time as an end'' \cite{Kant}. This principle resonates deeply in the fields of healthcare and technology, where a commitment to improving human welfare should guide innovations. The development of my application embodies this principle by seeking to enhance the lives of individuals undergoing oral immunotherapy. By providing a user-friendly platform for dose tracking, symptom monitoring, and educational resources, my project aims to empower patients to manage their condition and care effectively. In doing so, it not only promotes patients' physical well-being but also acknowledges their autonomy and agency in healthcare decision-making. 

Shifting gears to a utilitarian perspective, it is apparent that my application serves to maximize utility and happiness for the greatest number of people. John Stuart Mill famously posited, ``The creed which accepts as the foundation of morals, Utility, or the Greatest Happiness Principle, holds that actions are right in proportion as they tend to promote happiness, wrong as they tend to produce the reverse of happiness'' \cite{Mill}. From this standpoint, my project is justified by its potential to alleviate suffering and enhance the quality of life for a significant portion of the population affected by food allergies. Furthermore, as outlined in the Introduction, existing systems for logging OIT symptoms and doses rely on cumbersome handwritten logs and papers, causing frustration for both users managing the information and doctors analyzing it. Introducing a centralized digital platform to collect and aggregate this data not only maximizes happiness for patients by simplifying their experience, but also for healthcare providers, streamlining data management and analysis processes.

Drawing inspiration from contemporary philosopher Martha Nussbaum, my project also reflects a commitment to social justice by addressing disparities in healthcare access and information. As Nussbaum asserts, ``Central human capabilities provide the normative focus for political principles and public policy'' \cite{Nussbaum}. By facilitating access to vital resources and information through a mobile application, my project strives to promote equity in healthcare and empower individuals to advocate for their own well-being.

In essence, the ethical justification for my project extends beyond mere technological innovation – it embodies a commitment to compassion, autonomy, and social justice. By drawing upon philosophical insights on beneficence and social justice, my project seeks to contribute meaningfully to the advancement of healthcare and the promotion of human rights.

\section{Identification of Significant Ethical Issues}

The development and implementation of my application necessitate careful consideration of several significant ethical considerations. These encompass privacy and data security, informed consent, equitable access, and the accuracy and reliability of information. Each of these areas presents unique challenges and ethical dilemmas that must be addressed to ensure the ethical integrity of the application and the well-being of its users.

\subsection{Privacy and Data Security}
The issue of privacy and data security within the OIT application is multifaceted and demands careful deliberation. With the collection and storage of sensitive health information, users inherently entrust the application with personal data that can include medical history, allergy profiles, and treatment progress. This data is not only personal but also revealing of vulnerabilities and health conditions. Therefore, the responsibility falls on me, as the developer, to implement stringent security measures to safeguard this information against unauthorized access, breaches, or misuse.

Central to addressing privacy concerns is adherence to regulatory frameworks, such as the Health Insurance Portability and Accountability Act (HIPAA). Compliance with HIPAA standards ensures that patient confidentiality is upheld, and appropriate safeguards are in place to protect health information from unauthorized disclosure. This includes measures such as encryption, access controls, audit trails, and regular security audits to mitigate the risk of data breaches.

Moreover, transparency in data handling practices is paramount to maintaining user trust and confidence in the OIT application. Users should be provided with clear and accessible information about how their data is collected, stored, and used within the application. This includes details on data retention policies and procedures for accessing or deleting personal information. By empowering users with knowledge and control over their data, the application should foster a sense of agency and accountability in its privacy practices.

\subsection{Informed Consent}
Informed consent stands as a cornerstone of ethical practice in healthcare technology, as well as computing in general. The ACM Code of Ethics states, ``Computing professionals should establish transparent policies and procedures that allow individuals to understand what data is being collected and how it is being used, to give informed consent for automatic data collection, and to review, obtain, correct inaccuracies in, and delete their personal data'' \cite{ACM} This principle embodies the principle of respect for individuals' autonomy, acknowledging their right to make informed decisions about their participation in the use of the application and the sharing of their data.

When obtaining informed consent, it is vital that users are provided with comprehensive and comprehensible information about how their data will be collected, stored, and utilized within the application. This includes transparency regarding the purposes for which their data will be used, any potential risks or limitations associated with data sharing, and the mechanisms in place to safeguard their privacy and confidentiality. By arming users with this knowledge, they are better equipped to make informed decisions about their participation in the application and the extent to which they are comfortable sharing their personal information.

Furthermore, obtaining informed consent involves more than just providing information; it also requires ensuring that users have the capacity and opportunity to comprehend and deliberate upon the information presented to them. This is particularly relevant in the context of healthcare technology, where users may vary in their level of health literacy, technological proficiency, and cognitive abilities. As such, efforts should be made to employ clear language, visual aids, and accessible formats to facilitate understanding and decision-making.

Finally, the process of obtaining informed consent should be ongoing and iterative, rather than a one-time event. Users should be allowed to revisit and revise their consent preferences over time, as their circumstances, preferences, and understanding of the application evolve. This dynamic approach to informed consent not only respects users' autonomy but also acknowledges the fluid and evolving nature of their relationship with the application and their personal health information.

\subsection{Equitable Access}
Equitable access to my application is crucial to ensure that individuals of varying ages and technological backgrounds can benefit from its features. Age and digital literacy can significantly impact individuals' ability to access and utilize the application effectively. Given that many oral immunotherapy patients are young, and it's plausible that they, or their parents acting on their behalf, will engage with the application, accommodating varying levels of digital literacy becomes particularly pertinent in my project's design and usability. Another challenge lies in making the application accessible to diverse user groups, including those who may have limited access to technology or face barriers due to socioeconomic factors. 

For older adults or individuals with limited digital literacy, the application should feature intuitive interfaces, clear instructions, and user-friendly design elements to facilitate ease of use. Providing comprehensive user support, such as tutorials or helplines, can further assist users in navigating the application regardless of their technological proficiency. Furthermore, considerations of affordability and cost should be addressed to ensure that my application remains accessible to individuals across socioeconomic backgrounds. Offering the application free of charge can help mitigate financial barriers and ensure equitable access for all.

In summary, ensuring equitable access to my application requires proactive measures to address disparities in digital literacy, socioeconomic status, and more. By prioritizing user accessibility and inclusivity in design and implementation, the application can effectively reach and benefit individuals of diverse backgrounds and circumstances.

\subsection{Accuracy and Reliability of Information}
As mentioned in the Introduction, my application serves as an educational resource for users, providing information about food allergies, oral immunotherapy, and related topics. Ensuring the accuracy and reliability of the information presented is paramount to prevent misinformation and promote informed decision-making among users. Regular updates and validation of content by reputable sources are essential to maintain the integrity of the educational materials.

\section{Ethical Design Rationale}

In light of the significant ethical issues identified in the previous section, I made several decisions regarding the development and features of my application to mitigate ethical quandaries. These decisions were guided by principles of privacy protection, user empowerment, accessibility, and accuracy of information. Below, I elaborate on the rationale behind each decision.

\subsection{Integration with Apple Health and Other Apple Frameworks}

The decision to integrate my app with Apple Health and other Apple frameworks, such as Core Data, demonstrates my commitment to protecting user privacy and ensuring data security within my application. By integrating and aligning with Apple's ecosystem, which is renowned for its stringent privacy policies and robust security measures, my application adopts a proactive approach to safeguarding user data. Apple Health serves as a centralized repository for health-related information, allowing users to securely store and manage their medical data while maintaining control over its access and usage.

One of the key advantages of integrating with Apple Health is the assurance it provides users regarding the confidentiality and integrity of their sensitive health information. Apple has established itself as a trusted steward of user data, implementing encryption, access controls, and other advanced security mechanisms to protect against unauthorized access or breaches. By leveraging Apple's privacy infrastructure, my application can offer users a heightened level of confidence in the protection of their personal health data, fostering trust and credibility in its platform.

Additionally, integration with Core Data enhances the efficiency and reliability of data management within the OIT application. Core Data provides a robust and scalable solution for storing, querying, and manipulating application data, ensuring optimal performance and data integrity. By leveraging Core Data's capabilities, the application can streamline the data storage and retrieval processes while adhering to best practices for data security and privacy.

In essence, the decision to integrate with Apple Health and other Apple frameworks reflects a commitment to prioritizing user privacy and data security within the. By aligning with Apple's ecosystem and leveraging its established privacy infrastructure, the application can offer users a trusted and secure platform for managing their health information.

\subsection{Use of Emojis and Images for Enhanced Understanding}

I decided to incorporate emojis and images throughout my application due to its diverse user base, which has the potential to span across different ages and backgrounds. 

For users who may struggle with textual information due to language barriers, age, or limited literacy skills, emojis and images provide an alternative means of communication that is intuitive and easily understood. Emojis, in particular, convey emotions, actions, and concepts in a concise and visually appealing manner, allowing users to quickly interpret and relate to the content presented. By incorporating emojis strategically throughout the application, the app creates a more inclusive and user-friendly experience, catering to the needs of a diverse audience.

Moreover, the use of images within the application serves to complement textual information, reinforcing key concepts and enhancing comprehension. Visual representations can convey information more effectively than text alone, especially for complex or abstract ideas. By integrating images that illustrate concepts, procedures, or instructions, the application provides users with additional support in understanding and retaining information, regardless of their age or educational background.

\subsection{Inclusion of Tips to Guide Users}

The deliberate inclusion of tips within the OIT application represents a commitment to empowering users and enhancing their experience by offering valuable guidance and support. Recognizing the complexity of the application's features and functionalities, these tips serve as practical suggestions and insights aimed at helping new users navigate and utilize the application effectively.

The application's tips also work to promote informed decision-making and self-management of health among users. By providing users with actionable advice and recommendations, the application equips them with the knowledge and tools needed to make informed choices regarding their health and treatment journey. Whether it's tips on setting up the app, tracking progress, or adhering to treatment protocols, these insights enable users to maximize the benefits derived from the application and optimize their overall health outcomes.

Furthermore, the inclusion of tips reflects a user-centered approach to design and development, placing the needs and preferences of users at the forefront. By anticipating common challenges or questions that users may encounter, the application proactively addresses these concerns through timely and relevant tips. This proactive support not only enhances the user experience but also fosters a sense of confidence and autonomy among users in managing their health.

\subsection{Creation of a User Guide}

The creation of a comprehensive user guide for my application underscores a commitment to transparency, user empowerment, and informed consent. The user guide empowers users with the knowledge and understanding needed to navigate the application effectively. By offering detailed explanations of the application's features and functionalities, users are equipped to make informed choices about how they engage with the application and manage their health data. Moreover, the user guide serves as a reference tool, allowing users to access information about the application's capabilities and privacy practices at their convenience.

And, importantly, the user guide plays a critical role in facilitating informed consent among users. By providing transparency regarding data collection, storage, and utilization practices, the user guide ensures that users are fully aware of how their personal information is being managed within the application. This transparency enables users to make conscious decisions about their participation in the application and the extent to which they are comfortable sharing their data.

\subsection{Making the App Free and Available on the App Store}

The decision to offer my application free of charge and make it readily available on the App Store is driven by a commitment to promoting equitable access and ensuring that individuals from diverse socioeconomic backgrounds can benefit from its features without encountering any cost-related barriers. By offering the application for free, financial constraints are eliminated as a barrier to access, ensuring that individuals of varying economic means can avail themselves of its functionalities. This approach aligns with principles of social justice and inclusivity, ensuring that essential health resources are accessible to all, regardless of their ability to pay.

Additionally, making my application available on the App Store enhances its accessibility and reach, as the platform serves as a central hub for users to discover and download a wide range of applications. By leveraging the widespread availability and convenience of the App Store, the application can reach a broader audience and fulfill its mission of providing valuable health resources to as many individuals as possible.

Offering the application on the App Store also ensures compliance with platform guidelines and standards, further enhancing user trust and credibility. By adhering to App Store regulations, the application demonstrates a commitment to quality, security, and user privacy, thereby instilling confidence in users regarding the integrity and reliability of the platform.

In addition to promoting equitable access for individuals, offering the application free of charge also opens avenues for healthcare providers, including doctors' offices, to distribute the app to their patients. By removing the financial barrier associated with app acquisition, healthcare professionals can readily recommend and provide access to the OIT application as a supplementary tool for patient care.

And, for doctors' offices, the ability to offer the application to patients aligns with a patient-centered approach to healthcare delivery. Healthcare providers can leverage the application as a resource to complement traditional treatment methods, empowering patients to take a more active role in managing their health. By equipping patients with access to the application, doctors' offices can enhance patient education, facilitate treatment adherence, and promote better health outcomes.

Furthermore, the availability of the application free of charge enables doctors' offices to integrate it seamlessly into their existing workflows and patient care protocols. Healthcare professionals can incorporate the application into patient consultations, providing personalized recommendations and guidance tailored to individual treatment plans and goals. This integration fosters a collaborative approach to healthcare, where patients and providers work together towards shared treatment objectives.

As you can see, offering my application for free encourages widespread adoption and utilization among healthcare providers and their patients. By reducing barriers to access, doctors' offices can promote the use of the application as a standard component of care for patients undergoing oral immunotherapy. This widespread adoption not only enhances patient engagement and satisfaction but also facilitates data collection and monitoring, enabling healthcare providers to track patient progress and adjust treatment plans accordingly.

\subsection{Support for Different Font Sizes and Accessibility Features}

Finally, the decision to incorporate support for different font sizes and accessibility features within the application underscores a commitment to promoting inclusivity and ensuring that all users can effectively engage with the application, regardless of their individual accessibility needs.

By offering support for different font sizes, the application acknowledges the diverse range of users who may require adjustments to accommodate visual impairments or preferences. This customization empowers users to personalize their experience with the application, ensuring that they can comfortably read and interact with content without encountering any barriers or difficulties. Whether users require larger font sizes for improved readability or prefer smaller font sizes for increased information density, the application's flexibility ensures that their needs are accommodated.

In addition to supporting different font sizes, the application also incorporates other accessibility features to enhance usability for individuals with diverse needs. By prioritizing accessibility in its design and functionality, the application ensures that all users, regardless of their abilities or limitations, can access its content and features without encountering any accessibility-related barriers.