\chapter{Conclusion}

In conclusion, this project represents a significant step towards addressing the unmet needs of OIT patients through the development of an iPhone application. By delving into the challenges faced by individuals with life-threatening food allergies, the importance of innovative solutions is underscored. With a deeply personal connection to the subject matter, the project's overarching goal was established: to create an accessible, valuable tool that supports patients on their desensitization journey.

The comprehensive exploration of functional and non-functional requirements, detailed in Chapter 2, laid the foundation for the application's design. Key features such as integration with the Apple Health app, daily dose tracking, and intuitive data visualization were prioritized to ensure the app's usability and effectiveness. Moreover, the user-centric approach was further accentuated through the inclusion of educational resources, proactive notifications, and a keen focus on accessibility.

Throughout the project, numerous lessons were learned, ranging from technical intricacies to broader insights into patient needs and user experience. Notably, gaining proficiency in iOS development, understanding HealthKit permissions, and mastering data persistence techniques were among the technical learnings. Furthermore, insights into the challenges and nuances of managing food allergies, as well as the importance of empathy-driven design, significantly enriched the project's outcomes.

While the developed iPhone application represents a significant advancement in addressing the needs of OIT patients, there are still areas for improvement and refinement. Despite rigorous testing and user feedback, challenges such as ensuring seamless integration with diverse healthcare ecosystems and enhancing user engagement remain pertinent. Additionally, while the app strives for inclusivity, ongoing efforts are needed to address accessibility concerns and accommodate diverse user demographics effectively.

Looking ahead, future work could explore avenues for enhanced personalization, leveraging machine learning algorithms for predictive analytics, and fostering community engagement within the app. Furthermore, continued collaboration with healthcare professionals and allergy specialists can provide valuable insights for refining the app's features and expanding its impact. Ultimately, the journey towards improving the lives of OIT patients is ongoing, and this project serves as a pivotal step towards that endeavor.