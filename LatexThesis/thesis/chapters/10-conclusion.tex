\chapter{Conclusion}

In conclusion, this design report outlines a comprehensive solution to address the unmet needs of OIT patients through the development of an iPhone application. The backdrop of the challenges faced by individuals with life-threatening food allergies sets the stage for the critical importance of innovative solutions. With a deeply personal connection to the subject matter, the project goal is established – to create an accessible, valuable tool that supports patients on their journey towards desensitization.

The functional and nonfunctional requirements, outlined in Chapter 2, form the backbone of the application's design. Critical features, such as integration with the Apple Health app, daily dose tracking, and data visualization, are prioritized to ensure the app's usability and effectiveness. The user-centric approach is further emphasized through the incorporation of educational resources, notifications, and a focus on accessibility.

In essence, this design report represents a thorough and thoughtful approach to addressing the unmet needs of OIT patients. The envisioned iPhone application aims not only to facilitate daily tracking and monitoring but also to serve as a holistic support system, empowering users to manage their health with confidence and informed decision-making. As the project progresses into the implementation phase, the goal is to bring this vision to life, making a meaningful impact on the lives of those grappling with food allergies.

% \section{What I Learned}

% \section{Future Applications}