\chapter{Risk Analysis}

The development process inherently involves various risks that may impact the successful completion of the project. Identifying and analyzing these risks allows for effective mitigation strategies to be put in place.

The table below shows the risks that could potentially emerge during all stages of the project. Each row lists a risk, the chance of it occurring, its severity, the repercussions, and preventative measures.

\begin{table} [H]
    \centering
    \begin{tabular}{cccccc}
        \textbf{Risk} & \textbf{Probability} & \textbf{Severity} & \textbf{Impact} & \textbf{Consequences} & \textbf{Prevention}\\
         &  &  &  &  & \\
         &  &  &  &  & \\
         &  &  &  &  & \\
         &  &  &  &  & \\
         &  &  &  &  & \\
         &  &  &  &  & \\
    \end{tabular}
    \caption{Risk Analysis Table}
    \label{tab:risk_analysis}
\end{table}

% \section{Technical Risks}

% \subsection{Dependency on External APIs}

% \begin{itemize}
%     \item \textbf{Risk:} The application relies heavily on external APIs such as HealthKit, CareKit, and Swift Charts. Any changes, disruptions, or updates to these APIs may affect the functionality of the application.
%     \item \textbf{Impact:} Potential delays in development, features not working as intended, or a need for significant code adjustments.
%     \item \textbf{Mitigation:} Regularly monitor API updates, establish clear communication channels with API providers, and implement fallback mechanisms.
% \end{itemize}

% \subsection{Compatibility Issues}

% \begin{itemize}
%     \item \textbf{Risk:} The application must be compatible with various iPhone models and iOS versions. Changes in iOS updates or device specifications may lead to compatibility challenges.
%     \item \textbf{Impact:} User experience inconsistencies, potential crashes, or features not working on specific devices.
%     \item \textbf{Mitigation:} Regular testing on different devices and iOS versions, staying informed about upcoming iOS changes, and implementing backward compatibility where feasible.
% \end{itemize}

% \section{Security and Privacy Risks}

% \subsection{Health Data Privacy}

% \begin{itemize}
%     \item \textbf{Risk:} Handling sensitive health data introduces privacy concerns. Any breach or mishandling of this data could result in legal consequences and damage to the application's reputation.
%     \item \textbf{Impact:} Legal action, loss of user trust, and potential removal from the App Store.
%     \item \textbf{Mitigation:} Strict adherence to healthcare data privacy regulations (e.g., HIPAA), implementing robust encryption, and conducting regular security audits.
% \end{itemize}

% \section{Project Management Risks}

% \subsection{Time Constraints}

% \begin{itemize}
%     \item \textbf{Risk:} The project is subject to academic deadlines, and unforeseen delays in development may impact the timely completion of deliverables.
%     \item \textbf{Impact:} Academic repercussions, rushed development leading to potential bugs, and compromised quality.
%     \item \textbf{Mitigation:} Adhering to a well-defined project schedule, proactive time management, and regularly reassessing timelines.
% \end{itemize}

% \subsection{Single Developer Dependency}

% \begin{itemize}
%     \item \textbf{Risk:} The project is developed by a single individual, and any unforeseen events affecting the developer could disrupt the project.
%     \item \textbf{Impact:} Delays in development, potential knowledge gaps, and lack of continuity in case of emergencies.
%     \item \textbf{Mitigation:} Documentation of code and processes, regular backups, and collaboration with peers for code reviews and knowledge transfer.
% \end{itemize}

% \section{External Risks}

% \subsection{App Store Policies}