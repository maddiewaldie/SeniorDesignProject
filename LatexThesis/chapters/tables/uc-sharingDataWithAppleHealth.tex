\begin{table}[ht]
\centering
\caption{Use Case 02: Sharing Data with Apple Health}

\hspace{1em}
\renewcommand{\arraystretch}{1.7}

\begin{tabular}{|c|p{2em}|p{14cm}|}
\hline
\textbf{Dependencies} & \multicolumn{2}{|p{14cm}|}{User hasn’t enabled sharing data with Apple health yet} \\ 
\hline
\textbf{Description} & \multicolumn{2}{|p{14cm}|}{The system will behave in the following use case when the user enables sharing data with Apple health.} \\
\hline
\textbf{Precondition} & \multicolumn{2}{|p{14cm}|}{The user is going through the setup page.} \\
\hline
\multirow{6}{4em}{\textbf{Ordinary Sequence}} & \textbf{Step} & \textbf{Action} \\
& 1 & Application displays a screen with basic information for the user to fill out: Name, Birthdate, Allergens, Preferences, Share Data with Apple Health, Protect Application with FaceID \\
& 2 & User taps the “Share Data with Apple Health” toggle, toggling it on \\
& 3 & Application displays a pop up sheet, asking the user if they would like to share data with the Apple Health app. It will list out every symptom / thing that it will be reading from Apple Health / writing to Apple Health. \\
& 4 & User goes through all of the options and decides whether or not they would like to share that data, tapping on the toggles. \\
& 5 & User taps “Done” \\
\hline
\textbf{Postcondition} & \multicolumn{2}{|p{14cm}|}{The user will have all of the selected data shared with Apple Health.} \\
\hline
\textbf{Exceptions} & 0 & If the user has already gone through the onboarding process, they can edit their permissions in the Profile section, and then follow the ordinary sequence. \\
\hline
\textbf{Comments} & \multicolumn{2}{|p{14cm}|}{The user doesn’t have to select all of the categories, if they’re uncomfortable sharing that information.} \\
\hline
\end{tabular}
\end{table}
