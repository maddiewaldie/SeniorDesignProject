\chapter{Constraints and Standards}

Before delving into the development process, it is important to recognize and address the various constraints and standards that shape and govern the project, so that I can plan for them. This section delves into the constraints that have influenced the project's trajectory, ranging from time limitations imposed by academic schedules to legal considerations in ensuring the utmost user privacy. Simultaneously, the section explores the standards adhered to throughout the development process, aligning the project with recognized benchmarks in software engineering, accessibility, and interoperability. By understanding and adhering to these constraints and standards, my application will not only fit within the confines of practical realities but also conform to industry best practices, fostering a robust and ethically sound healthcare solution.

\section{Constraints}

Developing a software application as a solo endeavor brings its own set of challenges, and recognizing and navigating these constraints is paramount for success. While financial considerations are mitigated in this context, time and legal factors emerge as critical constraints that demand careful attention.

\begin{itemize}
    \item \textbf{Time:} The timeline of the senior capstone experience and academic deadlines impose significant time constraints on all phases of the project. Structuring the project plan and development process within these allocated timeframes is essential for meeting deliverables and achieving project goals.
    \item \textbf{Legal:} Legal considerations, particularly healthcare privacy regulations such as HIPAA, stand as significant factors influencing the project. Ensuring strict compliance with these regulations is non-negotiable to safeguard patient data and uphold legal standards. This legal awareness adds complexity to the development process, demanding meticulous attention to detail.
    \item \textbf{Maintenance Considerations:} Planning for the long-term maintenance of the project post-deployment is a critical aspect of the development process. Designing the system with foresight for future updates and enhancements ensures that ongoing maintenance can be executed seamlessly. This forward-thinking approach aims to maximize the application's lifespan and adaptability, addressing potential challenges before they arise.
\end{itemize}

In navigating these constraints, a strategic allocation of resources and a proactive approach to time management will be key to the success of the project. By addressing these challenges head-on, I hope to lay the foundation for a robust and sustainable software solution.

\section{Standards}

Ensuring that the developed application aligns with established industry standards is paramount for creating a robust, reliable, and inclusive healthcare solution. The adherence to specific standards across various facets of the project speaks to a commitment to excellence and user-centric design.

\begin{itemize}
    \item \textbf{IEEE Standards:} Adhering to a suite of IEEE standards has been integral to the development process. Particularly, in the realm of software engineering practices, these standards have acted as guiding principles. By aligning with IEEE standards, the project strives to meet industry benchmarks for quality and reliability, ensuring a strong foundation in electrical engineering and networking.
    \item \textbf{ISO Standards:} he application adheres to ISO standards, encompassing programming languages and design methodologies. Following ISO standards for languages, like Swift and SwiftUI, and design methodologies, such as UML, significantly contributes to enhancing the portability and compatibility of the application. This adherence reflects a commitment to creating a more robust and interoperable system.
    \item \textbf{Accessibility Standards:} Embracing industry design standards, the user interface has been meticulously crafted to adhere to best practices for the devices in use. The focus is on creating an intuitive and user-friendly interface that not only meets but exceeds accessibility standards. This design philosophy aims to make the application more inclusive for a diverse range of users.
    \item \textbf{Inclusivity Standards:} An emphasis on inclusivity is embedded in the project through the integration of accessibility libraries and tools. These additions ensure that the application is thoughtfully designed to be accessible to individuals with disabilities. This proactive step aligns with a broader commitment to creating a healthcare solution that is not just technologically advanced but also inherently inclusive and user-centric.
\end{itemize}