\chapter{Introduction}

\section{Background}

Every day, approximately 32 million Americans grapple with the challenges of food allergies [1]. A staggering 10 percent of the population finds itself on the precipice of a potentially life-threatening reaction with every meal. \par
Living with food allergies is a relentless battle – a constant state of vigilance. The mere act of eating transforms into a high-stakes game of Russian roulette, where the wrong choice can lead to dire consequences. Exclusion from social activities, bullying, and the looming specter of emergency room visits become haunting hallmarks of life for those afflicted [2]. Having lived with life-threatening, airborne food allergies my entire life, I know firsthand what this is like. In the first six months of my junior year of high school, I had over thirty-two allergic reactions, received seven epinephrine injections, was evacuated from school by ambulance four times, and was hospitalized once. The physical and emotional toll of living like this is immeasurable.\par
However, there is a glimmer of hope on the horizon – oral immunotherapy. Oral immunotherapy (OIT) is a medical treatment that gradually exposes a person to increasing amounts of an allergen to desensitize them to it. The goal of OIT is to help people with food allergies develop a tolerance to an allergen so that they can safely consume small amounts of it without experiencing an allergic reaction [3]. This groundbreaking approach represents a paradigm shift in the treatment of food allergies. Unlike the decades-old strategy of avoidance, oral immunotherapy offers the prospect of desensitizing patients to their allergens. By reaching a maintenance dose, a new level of protection allows them to be near, touch, or even consume foods they were previously forced to shun. This revolutionary approach has shown remarkable promise in clinical trials, raising the possibility of a life with fewer dietary restrictions and reduced fear of accidental allergen exposure.\par
Despite its immense potential, oral immunotherapy is not without its complexities and uncertainties – especially for the patient. It requires rigorous adherence to protocols, frequent medical supervision, and the emotional fortitude to face one's allergens head-on – not to mention, patients frequently experience side effects when updosing (i.e., increasing the intake of their allergen), such as itchiness or vomiting, and risk more severe complications, like anaphylaxis.\par
As with any medical treatment, patients need to log their doses and symptoms to keep track of their progress. But, for many, this is a difficult, impractical task. The magnitude of this challenge is evident in my three-inch binder filled with over 1,600 handwritten logs from my five-year OIT journey. And this binder will only grow, as I continue to take my maintenance dose of OIT for the rest of my life.\par
The lack of accessible tools to support patients throughout this journey is glaring, and it is within this context that the need for an innovative solution emerges. The overarching problem addressed by this thesis lies in the unmet needs of oral immunotherapy patients. These individuals require comprehensive support, guidance, and resources to navigate the intricate path toward desensitization safely and effectively. 

\section{Project Goal}

This project primarily focuses on the design and development of an iPhone application for OIT patients. By opting for the iOS platform, I aim to leverage Apple's robust privacy and security features, ensuring the safe storage of sensitive information. A key feature of the app will be its integration with the Apple Health app, allowing users to seamlessly view their oral immunotherapy data with other health metrics such as exercise and heart rate, all of which can significantly influence treatment outcomes. Integration with the Apple Health app will allow patients to easily, and securely, share their OIT information with their doctors. \par
Not only will the app facilitate daily dose tracking and symptom logging, but also it will provide the visualization of long-term trends, providing users with valuable insights into their progress. It will also serve as an educational resource, offering essential information about anaphylaxis and oral immunotherapy. Ultimately, my goal for this app is for it to become an accessible, valuable tool for anyone navigating this stressful, yet life-changing process.\par    

